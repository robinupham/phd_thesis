
% Uncomment to build this alone without subfiles:
% (also stuff at bottom)
\documentclass{scrbook}
% Koma script document options
\KOMAoption{paper}{a4}
\KOMAoption{fontsize}{11pt}
\KOMAoption{parskip}{half-} % paragraph spacing
% \KOMAoption{numbers}{enddot} % dot after section number
\KOMAoption{cleardoublepage}{plain} % include page numbers on blank pages
\KOMAoption{chapterprefix}{true} % 'Chapter' before number

% Packages
\usepackage{amsmath} % Gives \text command inside maths blocks
\usepackage{amssymb} % Various maths symbols
\usepackage{array} % Table formatting
\usepackage{bm} % Bold maths including Greek
\usepackage[format=plain]{caption} % Font sizing and alignment in captions
\usepackage{enumitem} % Allows numbering like 1.1 in ordered lists
% \usepackage{float} % Allows H placement of floats
\usepackage{graphicx}
\usepackage[hidelinks]{hyperref} % Hyperlinks without looking like it
% \usepackage{longtable} % Multi-page tables
% \usepackage{multicol} % For columns in text (not tables)
\usepackage{multirow} % For tables
\usepackage{neuralnetwork} % Neural net diagram
\usepackage{pdflscape} % Gives landscape environment
% \usepackage{scrlayer-scrpage} % To move page numbers
\usepackage{tabularx}
\usepackage{textcomp} % Added to fix \textasciiacute error on laptop
% \usepackage{tikz} % Diagrams (used for neural network example)
% \usepackage[pagenumberwidth=3em]{tocbasic}
% \usepackage{tocstyle} % ToC styling
\usepackage{upgreek} % Non-italic greek letters
\usepackage{xpatch} % Biblatex customisation

\usepackage[a4paper, inner=40mm, outer=15mm, top=30mm, bottom=30mm,footskip=15mm, headsep=15mm]{geometry}
% \usepackage[a4paper, inner=40mm, outer=30mm, top=50mm, bottom=50mm,footskip=20mm, headsep=20mm]{geometry} % footskip is space between footer (i.e. page number) and bottom of text
% min allowed is inner 40 mm, others 15 mm

\pagestyle{plain} % no header for front matter, overridden at end of front matter

% Caption setup
% \tablecaptionabove
\captionsetup[table]{labelsep=space}
% \captionsetup[table]{labelsep=space, skip=50pt, position=top}
\captionsetup[figure]{labelsep=space} % labelsep prevents dot followed by colon in captions

% Line spacing
\usepackage{setspace}
% \setstretch{1.4} % strangely this is > \onehalfspacing but < \doublespacing
\onehalfspacing
% \doublespacing

\raggedbottom % prevent huge spaces between paragraphs

% % % % % % % % % % % % % % % % % % % % % % % % %
% Font setup
% \usepackage{mathpazo} % Covers maths mode too
\usepackage[sc]{mathpazo} % Covers maths mode too, sc enables small caps
% \usepackage{palatino}
\usepackage[T1]{fontenc} % 8-bit font encoding
\addtokomafont{disposition}{\rmfamily} % Use serif throughout
% % % % % % % % % % % % % % % % % % % % % % % % %

% % % % % % % % % % % % % % % % % % % % % % % % %
% Section formatting setup
% \RedeclareSectionCommand[beforeskip=0pt]{chapter}
\RedeclareSectionCommand[beforeskip=0pt, innerskip=0pt]{chapter}
\RedeclareSectionCommand[beforeskip=10pt]{subsubsection}
\RedeclareSectionCommand[afterskip=1pt]{subsubsection}
% \setcounter{secnumdepth}{\subsubsectionnumdepth} % number up to subsubsections

% No dot after chapter number (https://tex.stackexchange.com/a/484727)
\renewcommand*{\chapterformat}{%
  \mbox{\chapappifchapterprefix{\nobreakspace}\thechapter
  \IfUsePrefixLine{}{\enskip}}%
}

% In the running header, separate chapter number and name with em dash
\renewcommand*{\chaptermarkformat}{%
\chapapp~\thechapter~---~}

% Create subsubsubsection below subsubsection but above paragraph, following https://tex.stackexchange.com/a/356574

\DeclareNewSectionCommand[
  style=section,
  counterwithin=subsubsection,
  afterskip=1pt,
  beforeskip=10pt,
  % afterskip=1.5ex plus .2ex,
  % beforeskip=3.25ex plus 1ex minus .2ex,
  % afterindent=false,
  level=\paragraphnumdepth,
  tocindent=10em,
  tocnumwidth=5em
]{subsubsubsection}
\setcounter{secnumdepth}{\subsubsubsectionnumdepth}
% \setcounter{tocdepth}{\subparagraphtocdepth}
\setcounter{tocdepth}{\subsubsubsectionnumdepth}

\RedeclareSectionCommands[
  level=\numexpr\subsubsubsectionnumdepth+1\relax,
  toclevel=\numexpr\subsubsubsectiontocdepth+1\relax,
  increaselevel,
]{paragraph,subparagraph}
\RedeclareSectionCommand[
  counterwithin=subsubsubsection,
  tocindent=12em,
  tocnumwidth=6em,
  beforeskip=10pt,
  afterskip=1pt, % line break after paragraph title
]{paragraph}
\RedeclareSectionCommand[
  tocindent=14em,
  tocnumwidth=7em,
  beforeskip=0pt
]{subparagraph}
% % % % % % % % % % % % % % % % % % % % % % % % %

% Autoref capitalisation
\def\chapterautorefname{Chapter}
\def\sectionautorefname{Section}
\def\subsectionautorefname{Section}
\def\subsubsectionautorefname{Section}

% % % % % % % % % % % % % % % % % % % % % % % % %
% Bibliography setup
\usepackage[backend=biber,
    % style=authoryear,
    style=authoryear-comp, % Don't repeat same author(s) in multiple citations
    giveninits=true,
    useprefix=true, % 'van der' etc.
    url=false,
    doi=false,
    isbn=false,
    eprint=false,
    uniquename=false, % Don't add initials in citation to disambiguate between authors with the same surname
    uniquelist=false, % Don't disambiguate in citation between different 'et al.' teams
    maxbibnames=10,
    minbibnames=10,
    maxcitenames=3,%  # 2,
    natbib, % Gives citep and citet commands
    labelalpha=true, % Use an 'alpha' label for each bib entry
    maxalphanames=1, % Use first author as the alpha label
    sorting=anyvt, % Sort by alpha (first author) then year
    block=par, % New line between 'blocks' of the bib entry
    dashed=false, % Reprint author list for each publication in bibliography
    sortcites=false % Show citations in the order supplied
]{biblatex}

% Citation/reference parameters
\renewcommand*{\nameyeardelim}{\addspace} % Space between author and year rather than comma
\renewcommand*{\finalnamedelim}{\addspace\&\addspace} % Ampersand rather than 'and'
\xpatchbibmacro{name:andothers}{{\finalandcomma}}{\addspace}{}{} % Space before 'et al.' rather than comma

% Citation-specific parameters
\DeclareCiteCommand{\blindcite}{\unspace}{}{}{\mancite} % Easy manual citations

% Reference-specific parameters
\AtEveryBibitem{\clearfield{title}} % Suppress title
\AtEveryBibitem{\clearfield{month}} % Suppress month
\DeclareNameAlias{author}{family-given} % Surname first for not just the first author
\DeclareNameAlias{editor}{family-given} % Same for editors
\renewbibmacro{in:}{} % Remove 'In:'
\DeclareFieldFormat{journaltitle}{#1} % Journal title in normal font rather than italics
\renewbibmacro*{volume+number+eid}{\printfield{volume}\printfield{number}\setunit{\addcomma\space}\printfield{eid}} % No dot after issue
\DeclareFieldFormat[article]{number}{\mkbibparens{#1}} % Volume in brackets
\DefineBibliographyStrings{english}{page = {}, pages = {}} % Suppress 'p.'/'pp.'
\renewbibmacro*{date+extradate}{\printtext{\printfield{year}\addcomma}} % Year not in brackets
\DeclareFieldFormat{pages}{\mkfirstpage[{\mkpageprefix[bookpagination]}]{#1}} % Only give starting page
\DeclareFieldFormat{url}{\url{#1}} % No 'URL' before URLs
% \renewcommand{\finentrypunct}{} % Remove final full stop
\renewcommand*{\newunitpunct}{\addcomma\space} % Commas between elements of bibitems

\DeclareBibliographyDriver{book}{%
  \printnames{author}%
  \space
  \printfield{year}%
  \newunit\newblock
  \printfield{booktitle}%
  \newunit
  , \printlist{publisher}%
\finentry}

\DeclareBibliographyDriver{inproceedings}{%
  \printnames{author}%
  \space
  \printfield{year}%
  \newunit\newblock
  \printfield{booktitle}%
  \newunit
  \printfield{volume}%
  \newunit
  \printfield{pages}%
\finentry}

\DeclareBibliographyDriver{incollection}{%
  \printnames{author}%
  \space
  \printfield{year}%
  \newunit\newblock
  \printfield{booktitle}%
  \newunit
  , ed. \printnames{editor},%
  \newunit\newblock
  \printlist{publisher}%
\finentry}

\DeclareBibliographyDriver{misc}{%
  \printnames{author}%
  \space
  \printfield{year}%
  \newunit\newblock
  \printfield{title}%
  \newunit
  \printfield{url}%
\finentry}

\addbibresource{refs.bib}
% % % % % % % % % % % % % % % % % % % % % % % % %

% Footnote spacing
% \deffootnote[1em]{1.5em}{1em}{\textsuperscript{\thefootnotemark~}}
\deffootnote[1em]{1em}{1em}{\textsuperscript{\thefootnotemark~}}

% Testing setting all penalties to zero
\binoppenalty=0
\brokenpenalty=0
\clubpenalty=0
\displaywidowpenalty=0
\exhyphenpenalty=0
\floatingpenalty=0
\hyphenpenalty=0
\interlinepenalty=0
% \linepenalty=0 % allowing this to be zero splits titles in a strange way
\postdisplaypenalty=0
\predisplaypenalty=0
\relpenalty=0
\widowpenalty=0

% Shorthands (non-Maths)
\newcommand{\lcdm}{$\Lambda$CDM}
\newcommand{\wcdm}{$w$CDM}
\newcommand{\Euclid}{\textit{Euclid}}
\newcommand{\Planck}{\textit{Planck}}
\newcommand{\Pcl}{Pseudo-$C_\ell$}
\newcommand{\pcl}{pseudo-$C_\ell$}
\newcommand{\ttp}{3$\times$2\,pt}

% Maths shorthands
\newcommand{\alm}{a_{\ell m}}
\newcommand{\Cl}{C_\ell}
\newcommand{\fsky}{f_\text{sky}}
\newcommand{\lmax}{\ell_\text{max}}
\newcommand{\lmin}{\ell_\text{min}}
\newcommand{\leff}{\ell_\text{eff}}
\newcommand{\tmin}{\theta_\text{min}}
\newcommand{\mathbfit}[1]{\bm{\mathit{#1}}}
\newcommand{\mathbfss}[1]{\bm{\mathsf{#1}}} % to match MNRAS \mathbfss
\renewcommand{\Re}{\operatorname{Re}}
\renewcommand{\Im}{\operatorname{Im}}

% ΛCDM parameters (maths mode)
\newcommand{\wo}{w_0}
\newcommand{\wa}{w_a}
\newcommand{\omm}{\Omega_\text{m}}
\newcommand{\omb}{\Omega_\text{b}}
\newcommand{\omc}{\Omega_\text{c}}
\newcommand{\sie}{\sigma_8}

% % Editing only
% \usepackage{xcolor}
% \newcommand{\todo}[1]{\textbf{{\color{red}{#1}}}}


% \usepackage{subfiles} % Best to do this last apparently

\pagestyle{headings}
\setcounter{chapter}{0} % deliberately 1 too low
\begin{document}

% Uncomment to use subfiles:
% \documentclass[../Thesis.tex]{subfiles}
% \begin{document}

\chapter{Cosmology}

Cosmology is the study of the Universe, the components that make it up, and its past and future evolution. Our understanding of these topics has improved greatly over the past century, beginning with the development of the theory of general relativity (GR), which describes gravity as an emergent phenomenon resulting from the effect of mass on the geometry of spacetime \citep{Einstein1916}. Solutions to GR were later developed \citep{Friedmann1922, Friedmann1924, Lemaitre1927, Lemaitre1933, Robertson1935, Robertson1936a, Robertson1936b, Walker1937} to describe an expanding universe that is homogeneous and isotropic. Meanwhile, observational evidence for universal expansion was provided by the discovery of `Hubble's law', which states that distant galaxies are moving away from Earth at a speed proportional to their distance \citep{Lemaitre1927, Hubble1929}.

The application of time symmetry to the expanding Universe implies that it has a finite age, and that the early Universe was extremely dense. This became known as the `Big Bang' model. An important prediction of the model is the existence of the cosmic microwave background radiation (CMB; see \autoref{Sec:cmb_intro}) \citep{Gamow1948a, Gamow1948b, Alpher1948a, Alpher1948b}, which was discovered in 1965 \citep{Penzias1965, Dicke1965}. Later observations of the CMB with the Cosmic Background Explorer (COBE) satellite confirmed that it has an almost perfect blackbody spectrum \citep{Mather1990, Mather1994, Fixsen1996}, providing perhaps the strongest evidence for the Big Bang model.

Meanwhile, it was deduced from observations of the orbital speeds of stars in nearby galaxies that an additional invisible `dark' form of matter must exist \citep{Kapteyn1922, Zwicky1933, Zwicky1937, Rubin1970, Freeman1970, Bosma1978}. It was later found that dark matter must be non-relativistic (`cold') and that it dominates the matter content of the Universe \citep{Rubin1980, Bond1982, Blumenthal1982, Davis1985}.

Since the Universe was known both to be expanding and to contain large amounts of gravitationally attractive matter, it was broadly expected that the expansion should be decelerating \citep[e.g.][]{Yoshii1988, Gasperini1993}. However, measurements of the deceleration rate derived from observations of distant supernova at the end of the last century revealed that the expansion of the Universe is in fact accelerating \citep{Riess1998, Perlmutter1999}. This was the most recent major change in our understanding of the Universe, and led to the development of the current standard model of cosmology (see \autoref{Sec:std_model} below) and to the era widely described as that of `precision cosmology' \citep[e.g.][]{Steigman2007}, though this perhaps belies the fact that the nature of so much of the Universe remains a mystery (see \autoref{Sec:open_questions}).

\section{Standard cosmological model}
\label{Sec:std_model}

The simplest cosmological model that is broadly consistent with all observations to date is the \lcdm{} model. $\Lambda$ represents the cosmological constant, which is an additional contribution to the energy density of the Universe that is constant with space and time---the simplest model of the `dark energy' required to account for the observed accelerating expansion. CDM stands for cold dark matter. These two dominant components of the Universe, dark energy and dark matter, constitute a respective 68.3\% and 26.8\% of the energy of the Universe. The remainder is mostly `baryonic' matter---normal matter as we know it on Earth, with very small contributions from the CMB and cosmic neutrinos.

The \lcdm{} Universe evolves following GR and its solutions for an expanding Universe that is homogeneous and isotropic on large scales, which are contained in the Friedmann equations \citep{Friedmann1922,Friedmann1924}. It is spatially flat, obeying Euclidean geometry. The remainder of the model is fitted to a minimal number of six parameters. However, the model can be extended by additional parameters to describe more complex behaviour, such as spatial curvature and time-dependent dark energy.

\subsection{Dark energy}

In \lcdm{} and its extensions, the components of the Universe are modelled as perfect fluids that can be characterised completely by their energy density $\rho$ and pressure $p$. The ratio of pressure to density is the dimensionless equation of state parameter $w$:
\begin{equation}
w = \frac{p}{\rho}.
\end{equation}
Unless otherwise specified, in this thesis $w$ will refer to the ratio of pressure to density for dark energy specifically.

If dark energy can be described by a cosmological constant, then
\begin{equation}
w = -1.
\end{equation}
The simplest extension is to allow $w$ to take a constant value other than -1. Alernatively, we can allow $w$ to vary with time, parametrised using the scale factor $a$, where $a = 0$ at the Big Bang and $a = 1$ today. Taylor expanding to linear order around a present value of $w_0$ gives
\begin{equation}
w \left( a \right) = w_0 + w_a \left( 1 - a \right),
\end{equation}
where
\begin{equation}
w_a = - \frac{dw}{da}.
\end{equation}

\section{Outstanding questions}
\label{Sec:open_questions}

Even if \lcdm{} is correct, it leaves many questions unanswered, including the physical nature of its dominant components. The physical form of dark matter is unknown: many theories exist, and current observations still permit a large range of different explanations \citep{Khimey2021}. Meanwhile, attempts to produce and directly detect dark matter in particle colliders have so far been unsuccessful \citep{Trevisani2018}. Even less is known of dark energy. It may be a constant vacuum energy, but the vacuum energy predicted by the standard model of particle physics is 120 orders of magnitude too large \citep[e.g][]{Carroll2001}. Alternatively it may be an additional scalar field, a class of theories known as `quintessence' \citep{Ratra1988, Caldwell1998, Carroll1998, Zlatev1999, Amendola2000}. Baryonic matter is comparatively well understood, but theories predict that there should be as much antimatter in the Universe as matter, contrary to all observations \citep{Sakharov1967, Steigman2018}.

Additionally, as the free parameters of the model are measured with increasing precision, tensions have emerged between values derived by different methods. The largest is the disagreement in the Hubble parameter $H_0$, which quantifies the current expansion rate of the Universe. $H_0$ can be directly estimated by measuring the redshifts of distant galaxies and determining their distance using `standard candles' of known brightness, typically Cepheid variables and Type IA supernovae. Alternatively, an estimate of $H_0$ can be derived from measurements of the cosmic microwave background. The two methods disagree, with a statistical significance of over $4 \sigma$ \citep{Planck2018VI, Riess2019, Verde2019}.

Finally, the model itself may simply be wrong. For example, it could be that one of its foundations, GR, is not correct, and that a modified theory of gravity is required---which may negate the need for dark energy to explain the accelerating expansion of the Universe \citep{Clifton2012, Joyce2016}. While \lcdm{} is able to explain all observations to date, it will continue to be tested with ever-increasing precision by a variety of methods, which will be described in the next section.

\section{Observational probes}

There are many observational methods by which we have learned and hope to continue to learn about the nature of the Universe, some of which are described in this section.

\subsection{Cosmic microwave background}
\label{Sec:cmb_intro}

In the early Universe (up to around 370\,000 years after the Big Bang), the temperature was so high that matter existed only in an ionised state, as separate protons and electrons, in a thermal equilibrium with photons. If an atom formed, it was reionised essentially immediately by a high-energy photon. As the Universe expanded, it cooled, and at a certain point photons no longer had sufficient energy to ionise matter. Since this point---known as recombination---the vast majority of these photons have travelled through the Universe with no or minimal interactions with other matter or radiation. They have continued to cool as the Universe has expanded, and today form the cosmic microwave background (CMB).

The CMB has very small anisotropies, due to quantum fluctuations in the early Universe. The \lcdm{} model predicts the distribution of these anisotropies depending on the values of the free parameters in the model. Successive satellite missions---COBE (1989--1993), the Wilkinson Microwave Anisotropy Probe (WMAP; 2001--2010), and \textit{Planck} (2009--2013)---have measured the anisotropies with increasing precision \citep{Bennett1996, Komatsu2011, Planck2015XIII}. The final cosmological analysis of the \textit{Planck} mission in 2018 provided the most precise cosmological constraints to date \citep{Planck2018VI}.

Some future CMB experiments are aimed at measuring its polarisation with increased precision, which may lead to a detection of the first clear evidence of cosmic inflation \citep{Kamionkowski2016}. Others aim to measure spectral distortions: deviations from a blackbody spectrum occuring due to interactions between CMB photons and other matter or radiation, which can potentially offer insight into all epochs of the Universe from recombination to the present day \citep{Chluba2021}.


Future CMB experiments are aimed at what exactly? inflation (B-modes), spectral distortions?

 without interacting

- where does the cmb come from

- as

and how they can relate to the outstanding questions

- cmb

- particle detectors detecting extraterrestrial particles, plus colliders

- strong lensing can differentiate between models of DM, maybe gravity too

- LSS other than lensing

- perhaps BAO or something should be mentioned, because really this section should be tied to a) how we know what we know and b) how we hope to know what we don't know

how we know what we know:

- expansion: supernovae

- more or less everything else lol: CMB

- but I think BAOs are relevant too

- yeah ok apparently the Planck 2018 results are CMB + SN + BAOs

...

- weak lensing, which gets a section to itself

\section{Weak gravitational lensing}


% Uncomment to build alone without subfiles:
\printbibliography[heading=bibintoc]
\end{document}
